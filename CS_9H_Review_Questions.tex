\documentclass[11pt]{article}
\usepackage{url,graphicx,tabularx,array,geometry,enumerate}
\setlength{\parskip}{1ex} %--skip lines between paragraphs
\setlength{\parindent}{0pt} %--don't indent paragraphs

% template modified from http://www.shawnlankton.com/2008/04/latex-template-for-homeworks/

%-- Commands for header
\renewcommand{\title}[1]{\large\textbf{#1}\\}
\renewcommand{\line}{\begin{tabularx}{\textwidth}{X>{\raggedleft}X}\hline\\\end{tabularx}\\[-0.5cm]}
\newcommand{\leftright}[2]{\begin{tabularx}{\textwidth}{X>{\raggedleft}X}#1%
& #2\\\end{tabularx}\\[-0.5cm]}

%\linespread{2} %-- Uncomment for Double Space
\begin{document}

% \title{Linguistics 100}
\leftright{May 5, 2015}{\textbf{CS Self-Paced Center}} %-- left and right positions in the header
\line

% title
\begin{center}
    \Large\textbf{CS 9H Final Exam Review Problems}\\
    \large{Python}
 \end{center}



\section{Warm-up}
What will Python 2 print?
\begin{enumerate}[a)]
\item \texttt{>>> (2**3, 3) + (1,2)}
\vspace{1cm}

\item \texttt{>>> "foo" and [] or ""}
\vspace{1cm}

\item \texttt{>>> "foo" and [] or 42}
\vspace{1cm}

\item \texttt{for i in "0123": print i*2}
\vspace{1cm}

\item Write a statement that returns \texttt{[1, 2, 4, 8, 16, 32]} (list of powers of 2 through $2^{5}$)
    \begin{enumerate}[i)]
    \item using \texttt{map} and \texttt{lambda}.
    \vspace{1cm}
    \item using list comprehensions.
    \vspace{1cm}
    \end{enumerate}
    
\item \begin{verbatim}
>>> def rearrange(arg): arg = arg[len(arg)//2:] + arg[:len(arg)//2]
>>> a = [2,4,6,8,10]; rearrange(a)
\end{verbatim}
\vspace{1cm}


\end{enumerate}

% File IO problem??

\section{Data Structures}
What does the following function do? What is the structure of the parameter it takes? Give a general explanation, don't just describe each line of code. Provide an example input and output.
\begin{verbatim}
def wat(foo):
    ret_val = {}
    for thing in foo:
        if thing[0] not in ret_val:
            ret_val[thing[0]] = {}
        for baz in thing[1]:
            if baz not in ret_val[thing[0]]:
                ret_val[thing[0]][baz] = thing[1][baz]
            else:
                ret_val[thing[0]][baz] += thing[1][baz]
    return ret_val
\end{verbatim}
\vspace{15cm}

\section{OOP}
\begin{enumerate}[a)]
\item Design a \texttt{Picture} class that holds a 2-dimensional collection of \texttt{Pixel} objects. Implement the following methods. The only error handling you need to worry about here is:
\begin{itemize}
\item When creating \texttt{Pixel} instances, the rgb values must be numbers [0,255]
\item when cropping a \texttt{Picture}, the new dimensions must be smaller than or equal to the current dimensions (can't crop a \texttt{Picture} to be larger).
\end{itemize}
In these cases, throw an appropriate type of \texttt{Exception} with a helpful message.
\begin{verbatim}
class Pixel:
    def __init__(self, r, g, b):




        self.r = 
        self.g = 
        self.b = 

    def __str__(self):
        return str((self.r, self.g, self.b))

class Picture:


    def __init__(self, pixel_array):
        self.pixels = 
        self.length = 
        self.width = 

    def __str__(self):
        repr = []
        for row in self.pixels:
            for pixel in row:
                repr.append(str(pixel) + "  ")
            repr.append("\n")
        return "".join(repr)

    def crop(self, new_length, new_width):
        """crop the Picture to have the new dimensions
        Ex. if a 3x3 image has pixels at
        0 1 2
        3 4 5
        6 7 9
        calling crop(2,2) would update the image to only contain
        0 1
        3 4"""










    def grayscale(self):
        """convert the Picture to grayscale by setting the RGB values
        of each Pixel to the average of the RGB values of the Pixel.
        Ex. if there is a pixel
        p = Pixel(100, 0, 20)        
        in a Picture img, after calling img.grayscale(), the value of p
        would be equivalent to Pixel(40, 40, 40)
        """










    def add_filter(self, filter_name, filter_function):




        
    def apply_filter(self, filter):
    
    
    
    
\end{verbatim}

\item You realize that with the endless possibilities of image processing, it would be useful to let programmers to create and use their own filters. Update \texttt{Picture} to be able to store and use arbitrary filters. Assume that filters do not take any parameter. Example interaction, assuming that \texttt{sharpen} has been defined correctly earlier:
\begin{verbatim}
>>> img1 = Picture([[Pixel(100,0,20)]*4]*3) 
>>> img2 = Picture([[Pixel(20,40, 60)]*5]*5) 
>>> img1.add_filter("sharpen", sharpen)
>>> img2.apply_filter("sharpen") # img2 has been sharpened
\end{verbatim}

\item What happens if you do
\begin{verbatim}
>>> pixels = [[Pixel(100,20,60)]*2]*2
>>> img1 = Picture(pixels)
>>> img2 = Picture(pixels)
>>> img1.grayscale()
>>> print img2

\end{verbatim}

\end{enumerate}

\section{So you think you know Python?}
\textnormal{These are just for fun, and are not topics covered on the final. The first is from programmingwats.tumblr.com and the second is via Mehrdad.}

\begin{verbatim}
>>> def my_append(item, lst = []):
>>>     lst.append(item)
>>>     return lst
>>> print my_append(1)
[1]
>>> print my_append(5, [3, 1, 4, 1])
[3, 1, 4, 1, 5]
>>> print my_append(1)
[1, 1]
\end{verbatim}


\begin{verbatim}

>>> foo = float('nan')
>>> foo in [foo]
True
>>> float('nan') in [foo]
False
>>> foo is foo
True
>>> foo == foo
False
\end{verbatim}

\end{document}
